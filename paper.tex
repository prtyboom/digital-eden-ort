\documentclass[11pt, a4paper]{article}

\usepackage[utf8]{inputenc}
\usepackage[english]{babel}
\usepackage{amsmath, amssymb}
\usepackage{graphicx}
\usepackage{geometry}
\usepackage{hyperref}
\usepackage{fancyhdr}

\geometry{top=2cm, bottom=2cm, left=2.5cm, right=2.5cm}
\pagestyle{fancy}
\fancyhf{}
\rhead{Digital Eden / ORT Simulation}
\lhead{Fedor Kapitanov}
\cfoot{\thepage}

\title{\textbf{Digital Eden: Computational Evidence for Phase Transitions in Consensus Reality}\\
\large Agent-Based Simulation of Ontological Resolution Theory}
\author{\textbf{Fedor Kapitanov} \\ Independent Researcher, Moscow \\ \texttt{prtyboom@gmail.com}}
\date{December 2025}

\begin{document}

\maketitle

\begin{abstract}
We present Digital Eden, an agent-based simulation testing the Ontological Resolution Theory (ORT). The model treats reality as a consensus field with high inertia. We demonstrate that a single high-coherence agent (the Operator), supported by 12 resonant nodes (Apostles), can trigger a phase transition in a field dominated by 10,000 low-value cells. The key finding: after the Operator termination at a critical pressure threshold (Gethsemane point), light propagates spontaneously through the field, raising the mean from 0.20 to 0.67, a 233 percent increase.
\end{abstract}

\section{Related Work}

\subsection{Opinion Dynamics}
Castellano et al. \cite{castellano2009} provide a comprehensive review of social dynamics models including the Voter model, Sznajd model, and bounded confidence models. Galam \cite{galam2012} shows that small committed minorities can overturn majority consensus.

\subsection{Leadership and Phase Transitions}
Kacperski and Holyst \cite{kacperski2000} demonstrated phase transitions in agent systems with leaders. Moscovici \cite{moscovici1976} provides empirical foundations: consistent minorities change majorities through behavioral coherence.

\subsection{Critical Phenomena}
Bak \cite{bak1987} shows that complex systems evolve toward critical states where small perturbations trigger avalanches. Scheffer et al. \cite{scheffer2009} identified early-warning signals for critical transitions.

\subsection{What is New}
Digital Eden uniquely combines: reality as weighted consensus field, coherence-over-quantity influence, termination as phase transition trigger, and post-termination autonomous propagation.

\section{Introduction}
Ontological Resolution Theory (ORT) posits that reality is actively resolved through coherent attention:
\begin{itemize}
    \item \textbf{The Absolute} corresponds to value 1 (pure potential).
    \item \textbf{Collapsed Reality} corresponds to value 0 (entropy).
    \item \textbf{Consensus Pressure} acts as drag toward the local mean.
\end{itemize}

\section{The Engine}
The simulation models a 1D field of N=10,000 cells evolving by:
\begin{equation}
    \varepsilon_i(t+1) = I \cdot \varepsilon_i(t) + (1 - I) \cdot S_i(t)
\end{equation}
where I=0.98 is inertia and S is the weighted signal from agents.

Key mechanics: Pressure (environment resists high values), Resonance (aligned agents amplify each other), Gethsemane Trigger (pressure exceeds threshold), Light Spreading (wave propagates after termination).

\section{Experiment}

\subsection{Initial Conditions}
\begin{itemize}
    \item Field: N=10,000 cells, initial value 0.2
    \item Crowd: 500 NPCs, Coherence=1, Worldview=0.2
    \item Operator: 1 agent, Coherence=50, Worldview=1.0
    \item Apostles: 12 agents, Coherence=5, Worldview=0.6
\end{itemize}

\subsection{Timeline}
\begin{enumerate}
    \item Steps 0-66: Operator at 1.0, pressure accumulates
    \item Step 67: Gethsemane triggered (P=3.03)
    \item Step 82: Operator terminated (H=28.4)
    \item Steps 83-1000: Light spreads autonomously
\end{enumerate}

\section{Results}
Results are shown in Figure 1.

\begin{figure}[h]
    \centering
    \includegraphics[width=0.9\textwidth]{gethsemane_result.png}
    \caption{Field evolution. Light spreads after termination.}
    \label{fig:results}
\end{figure}

\begin{center}
\begin{tabular}{|l|c|c|}
\hline
Metric & Initial & Final \\
\hline
Mean field & 0.200 & 0.666 \\
Cells above 0.5 & 0 & 8,203 \\
Cells above 0.8 & 0 & 3,054 \\
\hline
\end{tabular}
\end{center}

\section{Conclusion}
Digital Eden demonstrates:
\begin{enumerate}
    \item Single coherent signal overcomes 98 percent inertia
    \item Small aligned cluster provides critical amplification
    \item Termination triggers self-sustaining propagation
    \item Result: 233 percent increase in mean field value
\end{enumerate}

\noindent\textbf{Code:} \url{https://github.com/prtyboom/digital-eden-ort}

\begin{thebibliography}{99}
\bibitem{castellano2009} Castellano, C. et al. (2009). Statistical physics of social dynamics. \textit{Rev. Mod. Phys.}, 81, 591.
\bibitem{galam2012} Galam, S. (2012). \textit{Sociophysics}. Springer.
\bibitem{kacperski2000} Kacperski, K., Holyst, J. (2000). Phase transitions with leaders. \textit{Physica A}, 287, 631.
\bibitem{moscovici1976} Moscovici, S. (1976). \textit{Social Influence and Social Change}. Academic Press.
\bibitem{bak1987} Bak, P. et al. (1987). Self-organized criticality. \textit{PRL}, 59, 381.
\bibitem{scheffer2009} Scheffer, M. et al. (2009). Early-warning signals. \textit{Nature}, 461, 53.
\bibitem{stapp2007} Stapp, H. (2007). \textit{Mindful Universe}. Springer.
\bibitem{penrose1994} Penrose, R. (1994). \textit{Shadows of the Mind}. Oxford.
\end{thebibliography}

\end{document}